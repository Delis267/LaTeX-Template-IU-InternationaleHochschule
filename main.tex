\documentclass[11pt, a4paper, oneside, ngerman]{article}

%%
%% Pakete & Formalia IU-Internationale Hochschule
%%
\usepackage{enumitem}
\usepackage{tikz} % Zum Zeichenn
\usepackage[ngerman]{babel} % neue deutsche Trennungsregeln, etc
\usepackage[hidelinks]{hyperref} % Hyperlinks ohne Umrandungen
\usepackage{setspace} % Abstände zwischen Absätzen
\usepackage[left=2cm, right=2cm, top=2cm, bottom=2cm]{geometry} % Seitenränder
\setstretch{1.5} % 1,5 Zeilenabstand
\usepackage{lipsum}  % Dummy-Texte
\usepackage{titlesec} % define size for section headings
\usepackage[nohyperlinks, printonlyused]{acronym} % Abkürzungen
\usepackage[utf8]{inputenc} % korrekte Darstellung von Umlauten
\usepackage[T1]{fontenc} % enable hyphenation for languages with accented characters
\usepackage{array}
\usepackage{caption}
\usepackage{listings}
\usepackage{tabto}
\usepackage{adjustbox}
\usepackage{float}
\usepackage{capt-of}


%% Schriftarten und -größen
\usepackage{fontspec}
\setmainfont{Arial} % Hauptfont Arial (lualatex oder xelatex benötigt)
\titleformat{\section}{\normalfont\fontsize{12pt}{1.5}\bfseries}{\thesection}{1em}{} % Überschriften 1pt größer
\titleformat{\subsection}{\normalfont\fontsize{12pt}{1.5}\bfseries}{\thesubsection}{1em}{} % Überschriften 1pt größer
\titleformat{\subsubsection}{\normalfont\fontsize{12pt}{1.5}\bfseries}{\thesubsubsection}{1em}{} % Überschriften 1pt größer
\def\UrlFont{\rm} % Print URLs not in Typewriter Font
\renewcommand{\footnotesize}{\fontsize{10pt}{1.5pt}\selectfont}

% automatischens Einrücken von Absätzen verhindern
\usepackage{changepage}
\setlength{\parindent}{0pt}
% 6pt Abstand nur zwischen Absätzen
\setlength{\parskip}{6pt}{}

% Leerseite ohne Seitennummer, nächste Seite rechts
\newcommand{\blankpage}{
 \clearpage{\pagestyle{empty}\cleardoublepage}
}

% Grafiken einbinden
\usepackage{graphicx}
% center images and tables
\makeatletter
\g@addto@macro\@floatboxreset\centering
\makeatother


%% Informationen für die PDF-Datei
\hypersetup{
 pdfauthor={Vorname Nachname},
 pdftitle={Titel}, %% Titel der PDF-Datei
 pdfsubject={IU...},
 pdfkeywords={},
 pdfcreator={lualatex},
 pdfproducer={lualatex - Vorname Nachname},
}

%%
%% Bibliographie & Sondereinstellungen
%%
\usepackage[babel]{csquotes}
\usepackage[backend=biber, style=apa, pagetracker, apamaxprtauth=20 ]{biblatex}

%%
%%  Caption Setup
%%
% Abkürzungen für Abbildung und Tabelle verändern laut Zitierleitfaden
\addto\captionsngerman{
  \renewcommand{\figurename}{Abb.}
  \renewcommand{\tablename}{Tab.}
}
\captionsetup{
    justification=raggedright,
    singlelinecheck=false,
    labelsep=period,
    skip=6pt
}

%%
%% Source Code Setup im IntelliJ-lightblue-style
%%
\usepackage{xcolor}
\usepackage{fontspec}
\setmonofont{Courier New}
\definecolor{blue}{HTML}{3777E6}
\definecolor{grey}{HTML}{8C8C8C}
\definecolor{light-blue}{HTML}{6A84DB}
%\renewcommand{\lstlistingname}{Abb.}% Listing -> Abb.
\lstset{
  frame=trbl,
  showstringspaces=false,
  columns=flexible,
  basicstyle={\linespread{1}\small\ttfamily},
  numbers=left,                
  numbersep=5pt,   
  numberstyle=\tiny\color{gray},
  keywordstyle=\color{blue},
  commentstyle=\color{grey},
  stringstyle=\color{light-blue},
  breaklines=true,
  breakatwhitespace=false,
  tabsize=2,
  xleftmargin=10pt,
}
\lstdefinelanguage{json}{
    morestring=[b]",
    morecomment=[l]{//},
    morekeywords={true,false,null},
    sensitive=false,
}

\lstdefinelanguage{xml}{
    morestring=[b]",
    morecomment=[s]{<!--}{-->},
    morekeywords={xmlns,version},
    alsoletter=/<>-,
    moredelim=[s][\color{red}]{<}{\ },
    moredelim=[s][\color{red}]{</}{>},
    moredelim=[s][\color{red}]{<}{>},
    stringstyle=\color{blue},
    identifierstyle=\color{black},
    keywordstyle=\color{purple},
    commentstyle=\color{gray}
}

%% 
%% Eigene Befehle
%%
\newcommand{\tabellequelle}{}

% \beginn{apatable}{Titel}{Inhalt der Tabelle}{Spaltenformat}{Quelle}
% Label= Titel
% |p{3cm}|p{2.5cm}|p{8cm}| oder {|l|l|l|}
%%%
\newenvironment{apatable}[4]{%
  \renewcommand{\tabellequelle}{#4}
  \renewcommand{\arraystretch}{1.2} % Mehr Abstand innerhlab der Tabelle
  \begin{table}[h]
  \caption{#1}
  \label{#2}
  \raggedright % linksbündig statt zentriert
  \begin{tabular}{#3}
  \hline
}{%
  \end{tabular}
  \vspace{6pt}
  \caption*{Quelle: \tabellequelle}
  \end{table}
}

%%%
%%% Eigener Command für Codelistings:
%%% Codelistings werden ins Abbildungsverzeichnis aufgenommen und auch damit nummeriert.
% \codelisting
%  {Titel}
%  {label}
%  {filepath oder code}
%  {Quelle:}
%%%
\newcommand{\codelisting}[5]{%
  \noindent\captionof{figure}{#1} 
  \label{#2}
  \lstinputlisting[language=#4]{resources/code/#3}
  \vspace{-6pt}
  \noindent{{Quelle: }#5}
}

% \abbildung
%  {Titel}
%  {label}
%  {filepath oder code}
%  {Quelle:}
%%%
\newcommand{\abbildung}[4]{%
  \begin{figure}[H]
    \label{#2}
    \caption{#1}
    \adjustbox{max width=\textwidth}{
        \includegraphics{resources/img/#3}
    }
    \caption*{{Quelle: }#4}
  \end{figure}
}

% TODO Remove Comma after second to last author and ampersand
% https://tex.stackexchange.com/questions/670888/biblatex-apa-7-modification
\makeatletter
\renewcommand*{\apablx@ifrevnameappcomma}{\@secondoftwo}
\makeatother

\DefineBibliographyExtras{ngerman}{%
  \renewcommand*{\finalandcomma}{}%
}


%% Bibliographie einbinden
\addbibresource{resources/references.bib} % your bib file


%% ++++++++++++++++++++++++++++++++++++++++++
%% Dokument
%% ++++++++++++++++++++++++++++++++++++++++++
\begin{document}

\pagenumbering{Roman} % Uppercase Roman numerals
% \pagenumbering{gobble} % switch off page numbering
%% Titelseite
\def\usesf{}
%\let\usesf\sffamily % diese Zeile auskommentieren für normalen TeX Font
\newcommand{\HRule}{\rule{\linewidth}{0.1mm}} % Defines a new command for the horizontal lines,

\setlength{\unitlength}{1pt}

\begin{titlepage}

\vspace{-39pt}\hspace*{300pt}\includegraphics[width=.21\paperwidth]{resources/IU_Logo.png}

\begin{center}
\hbox{}
\vfill
{\usesf}

\HRule \\[0.4cm]
{\huge\bfseries Workbook/Hausarbeit \par}
Kurs: Algorithmen...\\[2mm]
DLBIADPS01...\\
\HRule \\[0.4cm]
\vskip 1cm


{\large\bfseries Vorname Nachname (IU...)\\}

Studiengang (B.SC.) \\
\today % TODO: Check Datum
\vskip 6cm
\begin{tabular}{p{5cm}l}
Tutor: Dr. name \\
\end{tabular}
\vfill
\end{center}


\end{titlepage}
%% Titelseite Ende % Titelblatt

%\newpage
%% ++++++++++++++++++++++++++++++++++++++++++
%% Verzeichnisse
%% ++++++++++++++++++++++++++++++++++++++++++
%\stepcounter{page}
%\setcounter{tocdepth}{3}
%\tableofcontents % Inhaltsverzeichnis

%% TODO: Einkommentieren falls Abbildungsverzeichnis benötigt
\blankpage
\phantomsection
\addcontentsline{toc}{section}{Abbildungsverzeichnis}
\listoffigures % Abbildungsverzeichnis

%% TODO: Einkommentieren falls Tabellenverzeichnis benötigt
\blankpage
\phantomsection
\listoftables % Tabellenverzeichnis
\addcontentsline{toc}{section}{Tabellenverzeichnis}

%\blankpage
%\phantomsection
%\addcontentsline{toc}{section}{Listingverzeichnis}
%\lstlistoflistings % CodeListingverzeichnis

%% TODO: Einkommentieren falls Abkürzungsverzeichnis benötigt
\blankpage
\section*{Abkürzungsverzeichnis}
\addcontentsline{toc}{section}{Abkürzungsverzeichnis}

% Abkürzungsverzeichnis zur verwendung mit \ac{]
\begin{acronym}

\acro{iu}[IU]{IU Internationale Hochschule}
\acro{csv}[CSV]{Comma-separated values}
\end{acronym}
\label{last-roman-page}% Save last page of this resources

%% ++++++++++++++++++++++++++++++++++++++++++
%% Hauptteil
%% ++++++++++++++++++++++++++++++++++++++++++
\blankpage
\pagenumbering{arabic}
\section{Testdatei}

Testdatei für die \ac{iu}

%APA-Table ist eine eigene Umgebung.
\begin{apatable}{Titel der Tabelle}{tab:labelname}{|l|l|l|}{Eigene Darstellung.}
    \textbf{Titel1} & \textbf{Titel2} & \textbf{Titel3} \\
    \hline
    TEST1  & $O(1)$ & $O(1)$ \\
    TEST2  & $O(1)$ & $O(k)$ \\
    TEST3  & $O(1)$ & $O(k)$ \\
    \hline
\end{apatable}

%Codelisting mit Python Syntax Highlight im Abbildungsverzeichnis
\codelisting
  {Pythoncode}
  {code:pythoncode}
  {testcode.py}
  {Python}
  {Eigene Darstellung.}

%Abbildung mit Titel und Quellenangabe
\abbildung
{IU-Logo}
{img:iu-logo}
{IU_Logo.png}
{Abbildung von...}

% HIER SEITEN EINFÜGEN
\pagebreak
%% ++++++++++++++++++++++++++++++++++++++++++
%% Literatur
%% ++++++++++++++++++++++++++++++++++++++++++
\blankpage

\phantomsection
% set bibliography name
\renewcommand{\bibname}{Literaturverzeichnis}
\addcontentsline{toc}{section}{\bibname}
% \nocite{*} % nur angeben, wenn auch nicht im Text zitierte Quellen 

\begin{flushleft}

\printbibliography[title=\bibname]
\end{flushleft}

% %% ++++++++++++++++++++++++++++++++++++++++++
% %% Anhang
% %% ++++++++++++++++++++++++++++++++++++++++++
% \blankpage
% \appendix
% \include{resources/appendix} % Anhang % TODO: Add Appendix?

\end{document}